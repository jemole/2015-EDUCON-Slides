%
% $Id: $
%
%
% Compilar a .pdf con LaTeX (pdflatex)
% Es necesario instalar Beamer (paquete latex-beamer en Debian)
%

%
% Gr�ficos:
% Los gr�ficos pueden suministrarse en PNG, JPG, TIF, PDF, MPS
% Los EPS deben convertirse a PDF (usar epstopdf)
%

\documentclass{beamer}
\usetheme{Warsaw}
%\usebackgroundtemplate{\includegraphics[width=\paperwidth]{format/libresoft-bg.png}}
%\usepackage[spanish]{babel}
\usepackage[latin1]{inputenc}
\usepackage{graphics}
\usepackage{amssymb} % Simbolos matematicos
\usepackage{url}

%\definecolor{libresoftgreen}{RGB}{162,190,43}
%\definecolor{libresoftblue}{RGB}{0,98,143}

%\setbeamercolor{titlelike}{bg=libresoftgreen}

%% Metadatos del PDF.
\hypersetup{
  pdftitle={Computer programming as an educational tool in the English classroom: a preliminary study},
  pdfauthor={J. Moreno-Le�n, Gregorio Robles},
  pdfcreator={GSyC/LibreSoft \\ Universidad Rey Juan Carlos},
  pdfproducer=PDFLaTeX,
  pdfsubject={Code to learn with Scratch. English as a second language.},
}
%%

\begin{document}

\title{Computer Programming as an Educational Tool in the English Classroom}
\subtitle{A Preliminary Study}
\institute{jesus.moreno@programamos.es, grex@gsyc.urjc.es \\
GSyC/Libresoft, Universidad Rey Juan Carlos}
\author{J. Moreno-Le�n, Gregorio Robles}
\date{EDUCON 2015, Tallinn, March 19\textsuperscript{th} 2015}

\frame{
\maketitle
\begin{center}
\includegraphics[width=2cm]{format/libresoft-logo}
\hspace{0.5cm}
\includegraphics[width=5cm]{format/gsyc-urjc}
\vspace{0.5cm}
\includegraphics[width=3cm]{format/emadrid.png}
\end{center}
}


% Si el titulo o el autor se quieren acortar para los pies de p�gina
% se pueden redefinir aqu�:
%\title{Titulo corto}
%\author{Autores abreviado}

%% LICENCIA DE REDISTRIBUCION DE LAS TRANSPAS
\frame{
~
\vspace{3cm}

\begin{flushright}
\includegraphics[width=2.2cm]{figs/by-sa}

{\tiny
(cc) 2015 J. Moreno-Le�n and Gregorio Robles\\
  Some rights reserved. This work licensed under Creative Commons \\
  Attribution-ShareAlike License. To view a copy of full license, see \\
  http://creativecommons.org/licenses/by-sa/3.0/ or write to \\
  Creative Commons, 559 Nathan Abbott Way, Stanford, \\
  California 94305, USA. \\
\ \\
Some of the figures have been taken from the Internet \\
Source, and author and licence if known, is specified. \\
For those images, \emph{fair use} applies.
}
\end{flushright}
}
%%

\section{EDUCON 2015 - eMadrid Session}


%--------------------------------------------------------
\usebackgroundtemplate{\includegraphics[width=18cm]{figs/AngryBirds.png}}

\begin{frame}
\frametitle{Goal of our paper}

\begin{center}
\Huge {\bf Does the use of programming in the English class improve students academic outcomes?}
\end{center}

\end{frame}


%-----------------------    ---------------------------------
\usebackgroundtemplate{\includegraphics[width=14cm]{figs/audience.png}}
% http://cdn.netrafic.com/wp-content/uploads/2011/02/audience.gif

\begin{frame}
\frametitle{Audience}

Who should/could be interested in this presentation?

\begin{itemize}
  \item Primary and Secondary teachers.
  \item Language educators.
  \item Students.
  \item Developers of programming learning tools.
\end{itemize} 

\end{frame}

\usebackgroundtemplate{}
%--------------------------------------------------------
\usebackgroundtemplate{\includegraphics[width=13cm,height=9.2cm]{figs/english.jpeg}}
\begin{frame}
\frametitle{English language skills of Spanish students}


  \begin{columns}[T]
    \begin{column}{1\textwidth}
     \begin{block}{European Survey on Language Competences}
\begin{itemize}
  \item Assessed the English language skills of 53,000 secondary students from 14 countries
  \item European Commission target: 50\% of students in CEFRL B1 or B2 level
  \item Spanish results in 2014: just 30\%
\end{itemize}
    \end{block}
    \end{column}
    
  \end{columns}
\vspace{\baselineskip}
\hfill{\Tiny Background picture: http://onesixstudio.com }
\end{frame}

\usebackgroundtemplate{}

%--------------------------------------------------------
\usebackgroundtemplate{\includegraphics[width=13cm,height=9.2cm]{figs/kids.jpeg}}
\begin{frame}
\frametitle{The study}


  \begin{columns}[T]
    \begin{column}{1\textwidth}
     \begin{block}{Code to learn English with Scratch}
\begin{itemize}
  \item Quasi-Experimental Design
  \item 65 students, 4\textsuperscript{th} and 5\textsuperscript{th} graders
  \item Control groups and experimental groups
  \item Pre and Post tests
  \item Surveys
\end{itemize}
    \end{block}
    \end{column}
    
  \end{columns}

\end{frame}

\usebackgroundtemplate{}

%--------------------------------------------------------
%\usebackgroundtemplate{\includegraphics[width=13cm,height=9.2cm]{figs/plugins.png}}
% background: http://25.media.tumblr.com/b83aa72682992ab34b8ce7e61c0cb7f9/tumblr_menxc7qcq61ryin08o1_r1_1280.jpg
\begin{frame}
\frametitle{Findings: Surveys}

  \begin{columns}[T]
    \begin{column}{0.5\textwidth}
    \begin{figure}[t!]
    
      \includegraphics[width=5.6cm,height=4.5cm]{figs/survey1.png}
    
    \label{fig:repetition1}
    \end{figure}
Scratch helped me to learn English     
    \end{column}
    \begin{column}{0.5\textwidth}
    \begin{figure}[t!]
    
      \includegraphics[width=5.6cm,height=4.5cm]{figs/survey2.png}
    
    \label{fig:repetition1}
    \end{figure} 
    Scratch made me want to learn more English
    \end{column}
  \end{columns}

\end{frame}

\usebackgroundtemplate{}

%--------------------------------------------------------
%\usebackgroundtemplate{\includegraphics[width=13cm]{figs/iceberg.jpg}}
% background: http://www.wim-network.org/wp-content/uploads/2012/04/iceberg.jpg

\begin{frame}
\frametitle{Findings: English knowledge}

\begin{table}
\begin{center}
  \begin{tabular}{ | l | c | c | }
   \hline
              & Experimental group & Control group \\ \hline\hline
    Initial test & 5.05 & 5.13 \\ \hline
    Final test & 7.7 & 7.55 \\ \hline
    Improvement & 2.65 & 2.42 \\ \hline
  \end{tabular}
\end{center}
\caption{Pre-test and Post-test results}
\label{table:results}
\end{table}
\end{frame}

%--------------------------------------------------------
\begin{frame}
\frametitle{Findings: Teacher training}

  \begin{columns}[T]
    \begin{column}{0.5\textwidth}
 
     \begin{block}{Experimental group 1}
\begin{figure}[t!]
\begin{center}
\includegraphics[width=5.4cm,height=5.4cm]{figs/chart1.png}
\end{center}
Teacher with coding experience
\label{fig:results1}
\end{figure}
     \end{block}
    \end{column}
    \begin{column}{0.5\textwidth}
     \begin{block}{Experimental group 2}
\begin{figure}[t!]
\begin{center}
\includegraphics[width=5.4cm,height=5.4cm]{figs/chart2.png}
\end{center}
First-time programmer teacher
\label{fig:results2}
\end{figure}
     \end{block}

    \end{column}
  \end{columns}

\end{frame}

%--------------------------------------------------------
\begin{frame}
\frametitle{Findings: Coding skills}

\begin{figure}[t!]
\begin{center}
\includegraphics[width=11cm, height=5.5cm]{figs/mastery.png}
\end{center}
\label{fig:naming}
\end{figure}

\begin{center}
Computational Thinking Score - Dr. Scratch results 
http://drscratch.programamos.es
\end{center}
\end{frame}

%--------------------------------------------------------
%\usebackgroundtemplate{\includegraphics[width=13cm]{figs/take-away.jpg}}
% background: http://2.bp.blogspot.com/-78Eh4TBpdtU/UPw7ULV73PI/AAAAAAAAHAE/6DQfvPNCo-Y/s1600/8723052-stylized-red-stamp-showing-the-term-take-away-all-on-white-background.jpg
\usebackgroundtemplate{\includegraphics[width=13cm]{figs/future.png}}

\begin{frame}
\frametitle{Future Work}

\begin{enumerate}
  \item Increase the number of students
  \item Increase the type of subjects 
  \item Increase the working time of programming
\end{enumerate}
\vspace{\baselineskip}
\vspace{\baselineskip}
\hfill{\Tiny Background picture: Simon Cunningham }

\end{frame}

%--------------------------------------------------------
%\begin{frame}
%\frametitle{GSyC/LibreSoft}

%\begin{figure}[t!]
%\begin{center}
%\includegraphics[width=11cm]{figs/libresoft.jpg}
%\end{center}
%\label{fig:libresoft}
%\end{figure}

%\begin{center}
%The GSyC/LibreSoft research team at URJC (Madrid).
%\end{center}

%\end{frame}

%--------------------------------------------------------
%\begin{frame}
%\frametitle{Our work at GSyC/LibreSoft}

%\begin{figure}[t!]
%\begin{center}
%\includegraphics[width=2.5cm]{figs/research}
% http://www.memphis.edu/crow/images/research_2.jpg
%\hspace{0.1cm}
%\includegraphics[width=2.5cm]{figs/teaching}
% http://2.bp.blogspot.com/_uxgwfriLwSo/TOWDr8IjaLI/AAAAAAAABLM/d7H-G5jIq-c/s1600/teaching.gif
%\hspace{0.1cm}
%\includegraphics[width=2.5cm]{figs/development}
% http://www.vidadigitalradio.com/wp-content/uploads/2009/04/hackers_cartoons.jpg
%\hspace{0.1cm}
%\includegraphics[width=2.5cm]{figs/promotion}
% http://bloggeate.com/wp-content/uploads/2011/04/como-promocionar-tu-blog.jpg
%\end{center}
%\label{fig:whatwedo}
%\end{figure}

%\begin{center}
%GSyC/LibreSoft's tasks: research, teaching, development, promotion of free software.

%\end{center}
%\end{frame}

\usebackgroundtemplate{}

%--------------------------------------------------------
\frame{
\maketitle
\begin{center}
\includegraphics[width=2cm]{format/libresoft-logo}
\hspace{0.5cm}
\includegraphics[width=5cm]{format/gsyc-urjc}
\vspace{0.5cm}
\includegraphics[width=3cm]{format/emadrid.png}
\end{center}
}

\end{document}
